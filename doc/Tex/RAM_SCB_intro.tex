The \textbf{R}ing current \textbf{A}tmosphere interactions \textbf{M}odel with \textbf{S}elf \textbf{C}onsistent magnetic field (\textbf{B}) is a unique code that combines a kinetic model of ring current plasma with a three dimensional force-balanced model of the terrestrial magnetic field.  The kinetic portion, RAM, solves the kinetic equation to yield the bounce-averaged distribution function as a function of azimuth, radial distance, energy and pitch angle for four ion species ($H^{+}$, $He^{+}$, $N^{+}$, and $O^{+}$) and, optionally, electrons. The code is also set up to easily accept new ion species. The domain is a circle in the Solar-Magnetic (SM) equatorial plane with a configurable radial span (2 to 6.5 $R_{E}$ by default).  It has an energy range of approximately 100\,$eV$ to 500\,$KeV$.  The 3-D force balanced magnetic field model, SCB, balances the $\textbf{J} \times \textbf{B}$ force with the divergence of the general pressure tensor to calculate the magnetic field configuration within its domain.  The domain ranges from near the Earth's surface, where the field is assumed dipolar, to the shell created by field lines passing through the SM equatorial plane at radial distances which extend past the RAM domain to ensure proper overlap.  The two codes work in tandem, with RAM providing anisotropic pressure to SCB and SCB returning the self-consistent magnetic field through which RAM plasma is advected.

RAM-SCB has grown from a research-grade code with limited options and static magnetic field (RAM) to a rich, highly configurable research and operations tool with a multitude of new physics and output products.  This manual provides a guide to users who want to learn how to install, configure, and execute RAM-SCB simulations.  While the code is designed to make these steps as straight-forward as possible, it is strongly recommended that users review the publications listed in the Bibliography to ensure a thorough understanding of the physics included in the model.  Additionally, all users are asked to review the terms of use.

\section{About This Manual}
Users who want to install and begin quickly should start at Chapter \ref{chp:quick}, which outlines the path from installation to simulation with little detail.  The installation process is discussed fully in Chapter \ref{chp:install}.  Instructions on performing simulations, as well as several example simulations, are given in Chapter \ref{chp:run}. An outline of using this code in the Space Weather Modeling Framework is found in Chapter \ref{chp:swmf}. Useful scripts included in the distribution are described, in brief, in Chapter \ref{chp:scripts}. Finally, a complete list of all param file commands is found in Chapter \ref{chp:param}.

\section{TERMS OF USE \& DISTRIBUTION POLICY}

Use of the RAM-SCB software implies agreement with the terms herein. RAM-SCB is open source software that has been developed at Los Alamos National Laboratory (LANL). \\

\noindent
Redistribution and use in source and binary forms, with or without modification, are permitted provided that the following conditions are met: \\

1. Redistributions of source code must retain the copyright notice, this list of conditions and the following disclaimer. \\

2. Redistributions in binary form must reproduce the copyright notice, this list of conditions and the following disclaimer in the documentation and/or other materials provided with the distribution. \\

3. Neither the name of Los Alamos National Security, LLC, Los Alamos National Laboratory, LANL, the U.S. Government, nor the names of its contributors may be used to endorse or promote products derived from this software without specific prior written permission. \\

\noindent
This software was produced under U.S. Government contract 89233218CNA000001 for Los Alamos National Laboratory (LANL), which is operated by Triad National Security, LLC for the U.S. Department of Energy/National Nuclear Security Administration.

All rights in the software are reserved by Triad National Security, LLC, and the U.S. Department of Energy/National Nuclear Security Administration. The Government is granted for itself and others acting on its behalf a nonexclusive, paid-up, irrevocable worldwide license in this material to reproduce, prepare derivative works, distribute copies to the public, perform publicly and display publicly, and to permit others to do so. NEITHER THE GOVERNMENT NOR TRIAD NATIONAL SECURITY, LLC MAKES ANY WARRANTY, EXPRESS OR IMPLIED, OR ASSUMES ANY LIABILITY FOR THE USE OF THIS SOFTWARE.  If software is modified to produce derivative works, such modified software should be clearly marked, so as not to confuse it with the version available from LANL.

Additionally, redistribution and use in source and binary forms, with or without modification, are permitted provided that the following conditions are met:
1. Redistributions of source code must retain the above copyright notice, this list of conditions and the following disclaimer.
2. Redistributions in binary form must reproduce the above copyright notice, this list of conditions and the following disclaimer in the documentation and/or other materials provided with the distribution.
3. Neither the name of Triad National Security, LLC, Los Alamos National Laboratory, LANL, the U.S. Government, nor the names of its contributors may be used to endorse or promote products derived from this software without specific prior written permission.

THIS SOFTWARE IS PROVIDED BY TRIAD NATIONAL SECURITY, LLC AND CONTRIBUTORS ``AS IS'' AND ANY EXPRESS OR IMPLIED WARRANTIES, INCLUDING, BUT NOT LIMITED TO, THE IMPLIED WARRANTIES OF MERCHANTABILITY AND FITNESS FOR A PARTICULAR PURPOSE ARE DISCLAIMED. IN NO EVENT SHALL TRIAD NATIONAL SECURITY, LLC OR CONTRIBUTORS BE LIABLE FOR ANY DIRECT, INDIRECT, INCIDENTAL, SPECIAL, EXEMPLARY, OR CONSEQUENTIAL DAMAGES (INCLUDING, BUT NOT LIMITED TO, PROCUREMENT OF SUBSTITUTE GOODS OR SERVICES; LOSS OF USE, DATA, OR PROFITS; OR BUSINESS INTERRUPTION) HOWEVER CAUSED AND ON ANY THEORY OF LIABILITY, WHETHER IN CONTRACT, STRICT LIABILITY, OR TORT (INCLUDING NEGLIGENCE OR OTHERWISE) ARISING IN ANY WAY OUT OF THE USE OF THIS SOFTWARE, EVEN IF ADVISED OF THE POSSIBILITY OF SUCH DAMAGE

The references below represent critical milestones for RAM-SCB. Please cite the appropriate work when using RAM-SCB to give the developers proper credit.


\begin{table}[ht]
  \centering
  \begin{tabular}{l|l}
    Citation & Information \\
    \hline
    \hline
    Jordanova et al. [1996] & First description of ring current model (RAM) using dipolar magnetic field \\
    \hline
    Jordanova et al. [2006] & First extension of RAM for non-dipolar magnetic field and coupling with SCB \\
    \hline
    Zaharia et al. [2006] & Description of SCB model and coupling with RAM \\
    \hline
    Jordanova et al. [2010] & Full description of RAM extension for non-dipolar magnetic field \\
    \hline
    Welling et al. [2011] & First full description of one-way coupling with SWMF \\
    \hline
    Welling et al. [2015] & Description of two-way coupling of RAM-SCB with SWMF \\
    \hline
    Engel et al. [2019] & Improved implementation of SCB and robust numerics \\
  \end{tabular}
\end{table}




