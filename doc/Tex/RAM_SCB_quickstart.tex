This section provides a bare-bones approach for getting started with RAM-SCB. The instructions are aimed at users familiar with Unix-like environments. Tutorial-like guides for each step can be found in other chapters.

\section{Installation}

Installation and compilation of RAM-SCB requires a Fortran compiler, Perl interpreter, an MPI library compiled with the preferred Fortran compiler, and several external libraries which have additional requirements (most notably a C compiler, typically standard on Unix-like systems.)  The required libraries are:

\begin{enumerate}
\item{\href{https://www.gnu.org/software/gsl}{GNU Scientific Library}}
\item{\href{http://www.unidata.ucar.edu/downloads/netcdf/ftp/netcdf-4.0.1.tar.gz}{Unidata's NetCDF library}}
\end{enumerate}

If these libraries are installed in non-standard locations, use environment variables to make them visible to RAM-SCB. The two variables to be set are {\tt GSLDIR} and {\tt NETCDFDIR}.

Installation of RAM-SCB is handled via the Config.pl script found in the installation directory. The {\tt -h} option will print help, but the user will almost exclusively use Config.pl as follows:

\begin{verbatim}
Config.pl -install -compiler=gfortran -mpi=openmpi -openmp -ncdf=${NETCDFDIR} -gsl={GSLDIR}
\end{verbatim}

Other compiler/MPI combinations are of course available, see Chapter \ref{chp:install}.

Finally, use GNU Make to compile and, if desired, test:

\begin{verbatim}
make
make test
\end{verbatim}

\section{Execution}

RAM-SCB is run from \textit{run directories}, create one via {\tt make}:

\begin{verbatim}
make rundir RUNDIR=~/desired_run_location
\end{verbatim}

Run directories are soft-linked to the installation.  This means you can move them freely and create multiple run directories for simultaneous simulations. A fresh run directory has everything you need to execute the code immediatetly, simply start the executable.

\begin{verbatim}
./ram_scb.exe
\end{verbatim}

To customize the simulation, edit the PARAM.in file.  Chapter \ref{chp:param} has a complete description of all parameters.  Additional input files required must be placed in either the {\tt input\_ram} or {\tt input\_scb} subdirectories; output is located in similarly named locations.

Output data come in two formats: simple ASCII and NetCDF files.  While there are a plethora of tools available, we recommend users try the Python-based tools discussed in Chapter \ref{chp:viz}.